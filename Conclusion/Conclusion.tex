%\cleardoublepage

\chapter{CONCLUSIONS}
\label{ch:Conclusion}

\section{Conclusions}
The work presented in this thesis began by developing equipment to experimentally measure external convection coefficients for spiral-helical surface water heat exchangers. These surface water heat exchangers are used to reject heat to, or extract heat from surface water bodies for space cooling, or space heating applications. Once the experimental equipment was developed, experiments were performed under heat rejection and heat extraction conditions. The experiments were performed in a 3 acre (1.2 ha) test pond, as well as a 15 ft (4.6 m) test pool. For heat rejection experiments, the equipment initially developed by \cite{Austin1998} and later modified by \cite{Hansen2011} was use to heat and circulate water in a closed loop. For the heat extraction experiments, a new experiment was developed which utilized two heat pumps with a nominal capacity of 4 tons (14 kW) to cool and circulate propylene glycol solution in a closed loop.

After the completion of the experiments, experimentally derived convection correlations for spiral-helical surface water heat exchangers were developed from parameter estimation techniques. These convection correlations are summarized in Section \ref{sec:Correlation:Conclusions}. These correlations have been shown to predict outside Nusselt number to within 28\% RMSE for heat rejection cases, and within 13\% RMSE for heat extraction cases. When implemented in a simulation to compare predicted heat transfer rates these correlations predict heat transfer rate to with 7\% RMSE for heat rejection cases, and within 5\% RMSE for heat extraction cases. 

From the convection correlations summarized in Section \ref{sec:Correlation:Conclusions}, design diagrams were developed for SWHE design purposes. For the case of heat rejection, many different SWHE and SWHP system parameters were varied in a simulation to determine SWHE performance and sensitivity when compared to the ``typical" case. The parameters which were varied are: heat pump COP, lake temperature, horizontal/vertical tube-tube spacing, pipe size, pipe schedule, pipe material, circulating fluid flow rate, circulating fluid, internal and external fouling, and pond or quiescent surface water conditions. The design diagrams for space cooling are presented in Section \ref{sec:DesignTools:SpcCooling}, as well as the method for combining the percent differences for multiple parameters.

For space heating, a design diagram was given in Section \ref{sec:DesignTools:SpcHeating}, as well as some general recommendations for space heating application. As part of the design recommendaions for space heating applications, the tube-tube spacing should be kept as large as is feasible. This allows a margin of safety in case ice-on-coil conditions occur. Supplemental heating should also be considered when designing surface water heat pump systems for space heating applications.


\section{Recommendations for Future Work}
During the course of this work, several topics were identified which are classified as future research topics which were beyond the scope and time limits of this project. These topics are outlined in the following points.

\begin{enumerate}

	\item \textbf{Surface water cooling comparisons to conventional cooling systems.} A comparison of surface water cooling systems to conventional cooling systems is a topic which was identified while performing research for the publication \cite{MitchellSpitler2013}. Although there are many potential locations where these systems could be deployed, little information is available comparing the life cycle costs of these systems. This information is critical for a designer to seriously consider this design option.
	
	\item \textbf{SWHP system to GLHE system comparisons.} There is currently no information available to designers comparing the potential life cycle cost implications of choosing a SWHP system vs.\ a GLHE system, assuming that both options are available. This work would clarify the issues associated with ground heat storage, balanced heating and cooling loads, geographical and climatalogical effects.
	
	\item \textbf{Plate heat exchanger convection correlations in standing and flowing water.} There is currently little published information regarding the performance of surface water plate heat exchangers. These heat exchangers are an attractive alternative to spiral-helical or other HDPE SWHEs due to the reduced labor associated with their construction. These heat exchangers are used in a diverse range of operating conditions from stagnant ponds, flowing rivers, to coastal waters. \cite{Hansen2011} performed some work to determine overall heat exchanger performance, however interior convection resistance was roughly estimated. Some additional work to correlate plate interior and exterior convection coefficients would be useful for system designers and modelers.
	
	\item \textbf{Tesing using Copper SHWEs.} Because the thermal resistance of copper is inherently low, testing with copper SHWEs would eliminate the uncertainty in the external convection coefficient measurements. An alternative approach to the approach taken in this work would be to test using smaller copper SHWEs in a large tank. A method could then be developed to scale the heat transfer results to match actual size SWHEs.
	
	\item \textbf{Test and compare alternative surface water heat exchangers.} There is currently no work that compares the different varieties of surface water heat exchangers on common terms. Other varieties of surface water heat exchangers that are of interest are, slinky coils on the bottom, plate heat exchangers, and capillary type heat exchanegers.
	
\end{enumerate}
